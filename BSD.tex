\documentclass{amsart}
\usepackage{amssymb}
\usepackage{amsmath}
%\usepackage{url}
\usepackage{hyperref}

\newcommand{\software}[1]{\textsf{#1}} % same as defi currently
\newcommand{\Sage}{\software{SageMath}}
\newcommand{\Magma}{\software{Magma}{}}
\newcommand{\GP}{\software{PARI/GP}{}}
\font\cyr=wncyr10
\newcommand{\Sha}{\text{\cyr X}}

\newcommand\Q{\mathbb{Q}}
\newcommand\R{\mathbb{R}}
\newcommand\C{\mathbb{C}}
\newcommand\hh{\hat{h}}
\DeclareMathOperator{\tors}{tors}
\DeclareMathOperator{\real}{real}
\DeclareMathOperator{\complex}{complex}

\begin{document}

\title{The BSD formula over number fields}
\author{John E. Cremona}

\begin{abstract}
  We give an explicit statement of the Birch--Swinnerton-Dyer (BSD)
  formula for elliptic curves over general number fields, carefully
  defining all the terms and showing how they may be be computed using
  the software packages \Sage\ and \Magma.  In particular, we highlight
  certain differences in the way the formula has been stated in the
  literature.
\end{abstract}

\maketitle

The Birch--Swinnerton-Dyer (BSD) formula for elliptic curves over
\(\Q\) is clearly stated and explained in many texts, but for elliptic
curves over general number fields there are a number of subtleties
leading to the appearance of factors which are equal to \(1\) in the
rational case, and clear and explicit statements are harder to
find. We cite three, by Tate \cite{Tate}, Gross \cite{Gross}, and
T. Dokchitser \cite{Dok}, as well as a discussion on MathOverflow
\cite{MO} highlighting the common confusions which people have. The
point of this text is to clear up the potentially confusing issues.

For simplicity we concentrate on the case of elliptic curves, only
briefly mentioning the adjustments needed for Abelian varieties of
higher dimension \(g\).

Traditionally, the BSD formula for an elliptic curve~$E$ defined over
the number field~$K$ is written as a formula for the leading
coefficient of the L-function \(L(E/K,s)\) at \(s=1\), which we call
the \emph{L-value}, in terms of other quantities, including the order
of the \emph{Tate-Shafarevich group}~$\Sha(E/K)$. We will assume the
standard conjectures, first that \(L(E/K,s)\) has analytic
continuation so that its behaviour at \(s=1\) is defined, and also
that $\Sha$ is finite. Then we rearrange the formula so that it
expresses \(S=|\Sha|\) in terms of quantities which may be computed.

\section{Code}\label{code}

Where possible, we give the appropriate commands in \Sage\ (version
10.1) and \Magma\ (version V2.28-3) to compute these quantities. Many
of these functions have a precision parameter to control the precision
of the output: the input is always exact, but the output may in
principle be computed to arbitrary precision. \Magma\ can compute all
the quantities separately (with a little work) and can also evaluate
the entire formula. \Sage\ can compute all the quantities; unless
\(K=\Q\), computing the analytic rank and L-value itself currently
requires calling \Magma\ from \Sage, or using \GP\ library functions.

\subsection{\Magma}\label{magma}

For elliptic curves over general number fields all terms in the formula
are computed and combined in the function \texttt{ConjecturalSha()}:

\[
\texttt{ConjecturalSha(E,\ {[}P1,...,Pr{]});}
\]

\subsection{\Sage}\label{sage}

For elliptic curves over \(\Q\), we have:

\[
\texttt{E.sha().an\_numerical()}
\]

Work is under way to implement a similar function for elliptic curves
over number fields in future versions of \Sage.

\section{Notation}\label{notation}

\(K\) is a number field of degree \(d=r_1+2r_2\), where \(r_1\) and
\(r_2\) are the number of real and complex places, and discriminant
\(D_K\).

\(E\) is an elliptic curve defined over \(K\), given by an integral
Weierstrass equation or model with coefficients
\(a_1,a_2,a_3,a_4,a_6\).  The model is assumed to be integral, but not
minimal, as global minimal models do not always exist.  The
discriminant of the model is denoted~\(\Delta_E\), which is an
integral element of \(K\), and \(\mathfrak{d}_E\) denotes the minimal
discriminant ideal of \(E\), an integral ideal which is independent of
the model. Then \((\Delta_E) = \mathfrak{u}_E^{12}\mathfrak{d}_E\) for
some fractional ideal \(\mathfrak{u}_E\). A global minimal model
exists if and only if \(\mathfrak{u}\) is principal; in general the
class of \(\mathfrak{u}_E\) is an invariant of \(E\).

The Mordell-Weil group \(E(K)\) is a finitely-generated abelian group of
rank \(r=r(E(K))\), with finite torsion subgroup \(T=E(K)_{\tors}\). Let
\(P_1,\dots,P_r\in E(K)\) be points which generate \(E(K)/T\).

The L-function of~$E$ is $L(E/K, s)$.  By the weak BSD conjecture, we
assume that its order of vanishing at~$s=1$ (that is, the {\em
  analytic rank} of~$E/K$) is equal to the Mordell-Weil rank~$r$.

\section{Outline formula}\label{outline-formula}

The formula for \(S\) has the form
\[
S = \frac{(\text{field~factor})(\text{$L$-value})}{(\text{Mordell-Weil factor})(\text{local~factor})}.
\]
We will define each of these factors in turn.  Note that while
different sources disagree on the exact definitions of the factors,
the differences cancel out in the product: in particular there is a
factor $2^{r_2}$ which is included either by multiplying the
local-factor (as in \cite{Dok}) and or by dividing the field-factor
(as in \cite{Tate}).

\section{The field factor}\label{field-factor}

This is \[|D_K|^{1/2},\] so is equal to \(1\) for \(K=\Q\).

Gross includes this in with the local factors at infinite places.  In
some expositions, there is an extra factor of $2$ in the contribution
from each complex place to the local factor, and the field factor is
set to \(|D_K|^{1/2}/2^{r_2}\) to compensate.

For Abelian varieties of dimension \(g\) the factor is
\(|D_K|^{g/2}\).

\subsection{\Magma\ code:}\label{magma-1}

\[
\texttt{Sqrt(Abs(Discriminant(K)));}
\]

\subsection{\Sage\ code:}\label{sage-1}

\[
\texttt{RR(K.discriminant().abs()).sqrt()}
\]

\section{The L-value}\label{the-l-value}

This is \[L^{(r)}(E,1)/r!.\] Note that in general it is only conjectural
that the L-function of \(E\) has analytic continuation as far as
\(s=1\), so both the fact that this quantity is well-defined, and the
analytic properties assumed in the algorithm to evaluate its derivative
at \(s=1\), are conjectural.

\subsection{\Magma\ code:}\label{magma-2}

\[
\texttt{Evaluate(LSeries(E),1:\ Derivative:=r)\ /\ Factorial(r);}
\]

\subsection{\Sage\ code:}\label{sage-2}

For \(K=\Q\):
\[
\texttt{E.lseries().dokchitser().derivative(1,r)\ /\ factorial(r)}
\]

For general~$K$ (including $K=\Q$):
\[
\texttt{RR(pari(E).lfun(f`1+x+O(x\^{}\{r+1\})').polcoef(r))}\\
\]

\section{The Mordell-Weil factor}\label{mordell-weil-factor}

This is \[\frac{R_{NT}(P_1,\dots,P_r)}{|T|^2},\] where \(R_{NT}\) is
the N\'eron-Tate regulator, i.e.~the determinant of the N\'eron-Tate
height pairing matrix of the generators, whose \((i,j)\)-entry is
\[
\left\langle P_i,P_j\right\rangle = \frac{1}{2}(\hh(P_i+P_j) -
\hh(P_i) - \hh(P_j)),
\]
so in particular the diagonal entries are the canonical
heights~$\hh(P_i)$ of the \(P_i\).  Here the canonical height~$\hh$
needs to be normalised correctly:
\begin{enumerate}
\item It should be defined with respect to the \(x\)-coordinate, or
  equivalently with respect to the divisor~\(2(O)\). There is another
  convention (used in some of Silverman's papers, for example
  \cite{Silverman}) which is half of this, being defined with respect
  to the divisor \((O)\) instead of \(2(O)\). In the MathOverflow post
  \cite{MO}, Silverman explains why it makes sense to use \(2(O)\) in
  the context of BSD.
\item It should \textbf{not} be normalized (by dividing by the degree
  \(d\)) to be invariant under base-change. This non-normalised
  canonical height is known as the {\em N\'eron-Tate height}.  Note
  that both \Magma\ and \Sage\ normalize the height by
  default. \Sage\ has an option to not normalize canonical heights or
  the regulator, but with \Magma\ the regulator needs to be adjusted
  by multiplying by \(d^r\).
\end{enumerate}

\subsection{\Magma\ code:}\label{magma-3}

\[
\texttt{Regulator(P1,...,Pr)\ *\ d\^{}r\ /\ Order(TorsionSubgroup(E))\^{}2;}
\]

\subsection{\Sage\ code:}\label{sage-3}
Here the regulator function has a parameter to control the
normalisation, which by default is set to \textt{True}, so we
explicitly set it to \textt{False} for our purposes:
\[
\texttt{E.regulator\_of\_points({[}P1,...,Pr{]}, normalised=False)\ /\ E.torsion\_order()\^{}2}
\]

\subsection{Saturation}
The expressions given here will only give the correct regulator if the
points \(P_1,\dots,P_r\) are \emph{saturated}, i.e.~generate all of
\(E(K)\) modulo torsion, so that \([E(K):\left\langle
  P_1,\dots,P_r\right\rangle] = |T|\). An alternative expression is
sometimes seen which avoids this condition:
\[\frac{R(P_1,\dots,P_r)}{[E(K):\left\langle P_1,\dots,P_r\right\rangle]^2};\]
this is a neat trick, but not very useful in practice for computations.

Note that if we evaluate the formula with points are not saturated,
but only generate a subgroup (modulo torsion) of index \(n\), then
this factor, and hence the overall expression for \(S\), will be
\(1/n^2\) times the correct value. This will often lead to a
non-integral value of \(S\), which can be detected.

\subsection{Higher genus}\label{higher-genus}

For Abelian varieties \(A\) of general dimension \(g\) the formula
involves both \(A\) and its dual \(A'\), and points \(P_1',\dots,P_r'\)
generating \(A'(K)\) modulo torsion. (Recall that \(A\) and \(A'\) are
isogenous so have the same rank \(r\).) The factor is then
\[\frac{\det(\left\langle P_i,P'_j\right\rangle )}{\#A(K)_{\tors}\cdot \#A'(K)_{\tors}}\]
where now \(\left<\cdot,\cdot\right>\) is the height pairing between
\(A\) and \(A'\).

\section{The local factor}\label{local-factors}

The complete local factor is a product of individual local factors,
one for each place of \(K\). We will define the local factors at
infinite places in a way which depends on the model, but the complete
local factor also includes a correction factor making the complete
expression independent of the model.  The correction factor is~$1$ for
a global minimal model, but these do not always exist.

\subsection{Finite places}\label{at-finite-places}

At a finite place \(v\) the local factor is \(c_v\), the Tamagawa
number at \(v\), which is defined to be the index \(c_v =
[E(K_v):E(K_v)^0]\), i.e.~the number of connected components of
\(E(K_v)\), and is a positive integer. At a place of good reduction we
have \(c_v=1\), so the product of these factors over all finite places
is in effect a finite product.

\subsubsection{\Magma\ code:}\label{magma-4}

\[
\texttt{\&*TamagawaNumbers(E);}
\]
gives the product of the Tamagawa numbers \(c_v\).

\subsubsection{\Sage\ code:}\label{sage-4}

Over all number fields (including \(\Q\))
\[
\texttt{E.tamagawa\_product\_bsd()}
\]
gives the product of all Tamagawa numbers \(c_v\) multiplied by a
correction factor (defined below) to allow for the model not being
minimal, while \texttt{E.tamagawa\_product()} returns just the product
of the Tamagawa numbers.

\subsection{Infinite places}\label{at-infinite-places}

At each infinite place~$v$, the local factor is a suitably normalized period
\(\Omega_v\), and the total contribution from all infinite places is
\(\prod_{v\mid\infty}\Omega_v\). Theoretically the \(\Omega_v\) are
defined in terms of a N\'eron differential. We first define the periods
for a fixed model of \(E\) and then explain how to normalize them. Let
\(\omega_E = dx/(2y+a_1x+a_3)\) be the usual invariant differential
associated to a Weierstrass model for \(E\).

\subsubsection{Real places}\label{at-a-real-place}

If \(v\) is a real place we define \(\Omega_v(E)\) to be the integral
of \(\omega_E\) over \(E(\R)\). This is equal to the least positive
real period multiplied by the number of real components at the place
\(v\), which is \(1\) when \(v(\Delta_E)<0\) and \(2\) when
\(v(\Delta_E)>0\).

When \(v(\Delta_E)>0\) the period lattice of \(E\) at \(v\) has
generating periods \(x\), \(yi\) with \(x,y>0\), and \(\Omega_v=2x\).
When \(v(\Delta_E)<0\) the period lattice of \(E\) at \(v\) has
generating periods \(2x\), \(x+yi\) with \(x,y>0\), and \(\Omega_v=2x\).

\subsubsection{Complex places}\label{at-a-complex-place}

If \(v\) is a complex place we define \(\Omega_v(E)\) to be the integral
of \(\omega_E\wedge\overline{\omega_E}\) over \(E(\C)\). If the period
lattice of \(E\) at \(v\) has generating periods \(w_1\), \(w_2\) with
\(\Im(w_2/w_1)>0\), then \(\Omega_v=2\Im(\overline{w_1}w_2)\), which
is positive: it is double the covolume of the period lattice.

In \cite{Dok}, \(\Omega_v\) for complex \(v\) is defined to be double
the definition here, which appears to be an error.  Note that if we
instead defined $\Omega_v$ to be the lattice covolume, we would need
to compensate by dividing the field factor by~$2^{r_2}$.

\subsubsection{Adjustment for non-minimal models}\label{adjustment-for-non-minimal-models}

For a global minimal model, the minimal discriminant ideal is simply the
principal ideal \(\mathfrak{d}_E=(\Delta_E)\). In general,
\((\Delta_E)=\mathfrak{u}_E^{12}\mathfrak{d}_E\) for some fractional
ideal \(\mathfrak{u}_E\) whose valuation at each prime \(\mathfrak{p}\) is
\(1/12\) of the difference between the valuation at \(\mathfrak{p}\) of
\(\Delta_E\) and that of the local minimal model at \(\mathfrak{p}\).
Hence
\[
N(\mathfrak{u}_E) =  \left|\frac{N(\Delta_E)}{N(\mathfrak{d}_E}\right|^{1/12}.
\]

The product \(\prod_{v\mid\infty}\Omega_v N(\mathfrak{u}_E)\) is
independent of the model: scaling the model for~$E$ by~$\lambda\in
K^*$ multiplies~$\mathfrak{u}_E$ by the principal ideal~$(\lambda)$
and hence multiples~\(N(\mathfrak{u}_E)\) by \(|N(\lambda)|\); in
compensation, each~\(\Omega_v\) is divided by~$|\lambda|_v$, and the
product of these is also~\(|N(\lambda)|\).

\subsection{Complete local factor}\label{complete-local-factor}
The complete local factor, including a contribution from all finite
and infinite places and independent of the model,  is
\[
\prod_{v\nmid\infty}c_v\prod_{v\mid\infty}\Omega_v N(\mathfrak{u}_E)
=
\prod_{v\nmid\infty}c_v\prod_{v\mid\infty}\Omega_v \left|\frac{N(\Delta_E)}{N(\mathfrak{d}_E}\right|^{1/12}.
\]

\subsection{\Magma\ code:}\label{magma-5}
\[
\texttt{Periods(E);}
\]
gives a basis for the period lattice with respect to each infinite
place, starting with the real places. In evaluating
\texttt{ConjecturalSha()} the product of the actual real and complex
periods \(\Omega_v\) are computed from these, including factors of 2
for the real places where the discriminant is positive and a factor
\(2^{r_2}\) from the Dokchitser normalisation of the complex periods,
but the individual \(\Omega_v\) values are not easily accessible to
the user.

\subsection{\Sage\ code:}\label{sage-5}
For each infinite place,
\[
\texttt{E.period\_lattice(v).omega()}
\]
gives the value of \(\Omega_v\), including the factor of 2 for real
places where the discriminant is positive.  Hence the complete local
factor is given by
\begin{multline*}
  \texttt{prod({[}EK.period\_lattice(v).omega()\ for\ v\ in\ K.places(){]})}\\
    *\ \texttt{E.tamagawa\_product\_bsd()}
\end{multline*}

\section{The full formula}\label{the-full-formula}
The analytic order of $\Sha$ is given by the formula
\[
S =
|D_K|^{1/2}
\cdot
\frac{L^{(r)}(E,1)}{r!}
\cdot
\frac{|T|^2}{R(P_1,\dots,P_r)}
\cdot
\frac{1}{N(\mathfrak{u}_E)\prod_{v\nmid\infty}c_v\prod_{v\mid\infty}\Omega_v}.
\]

\section{Comparisons}\label{comparisons}

\subsection{Tate}\label{tate}

In \cite{Tate}, Tate has a factor \(|\mu|^g\) where \(g\) (which is
\(d\) in Tate's notation) is the dimension of the variety and he
states that \(|\mu|=|D_K|^{1/2}/2^{r_2}\).

The difference between this and our field factor can be accounted for
by doubling the contribution at the complex places, the normalization
for which Tate does not make explicit.

\subsection{Gross}\label{gross}

In \cite{Gross}, Gross states the formula for an abelian variety $A$
in the form
\[
\text{L-value} = M(A)R(A)h(A),
\]
where \(h(A)=|\Sha|\). In the analogy with the analytic class number
formula, $\Sha$ plays the role of the class group whose order is
usually denoted \(h\). The other two factors match our Mordell-Weil
factor, and the product of the field and local factors respectively.
In the case of an elliptic curve $A=E$, in our notation these are
\begin{itemize}
  \item
\(R(E)=R_{NT}(P_1,\dots,P_r)/|T|^2\); and
\item
\(M(E) = M_{\infty}(E)M_f(E)\), where
\[M_{\infty}(E) = \prod_{v\mid\infty}\Omega_v \cdot N(\mathfrak{u}_E) \cdot |D_K|^{-1/2},\]
and
\[
M_f(E) = \(\prod_{v\nmid\infty}c_v.
\]
\end{itemize}

\subsection{Dokchitser}\label{dokchitser}

In \cite[pp.~3--5]{Dok}, Dokchitser defines the regulator and L-value
as we have. He states that the regulator is the determinant of the
{N\'eron-Tate height-pairing}, where (as above) the prefix
``N\'eron-Tate'' indicates that the height is relative to \(K\) and
not the normalized absolute one.

Dokchitser combines all the local factors together into a quantity called
\(C_{E/K}\): Writing \(\omega\) for any invariant differential on \(E\)
(with arbitrary scaling), and \(\omega_v^o\) the local N\'eron
differential at a finite place \(v\), he defines \[
  C_{E/K} = \prod_{v\nmid\infty} c_v \left\vert\frac{\omega}{\omega_v^o}\right\vert_{_v}
    \>\cdot\>\prod_{{v|\infty},{\real}} \int\limits_{E(K_v)}\!\! |\omega|
    \>\cdot\>\prod_{{v|\infty},{\complex}} 2\!\!\!\int\limits_{E(K_v)}\!\! \omega_E\wedge \bar\omega_E,
\] with \(c_v\) the local Tamagawa number at \(v\) and \(|\cdot|_v\) the
normalised absolute value on \(K_v\). If \(\omega\) is the differential
associated to a global minimal model then
\(\left\vert\frac{\omega}{\omega_v^o}\right\vert_{_v}=1\) for all finite
\(v\) and this reduces to our unadjusted local factor. In general the
factor \(N(\mathfrak{u}_E)\) in our formula accounts for the difference in
scaling between the differential \(\omega_E\) and the local N\'eron
differential \(\omega_v^o\) at every place.

We see that \(C_{E/K}\) is independent of the model of \(E\), and is
$2^{r_2}$ times our local factor.  Otherwise, his formula is identical
to ours.

\section{Origin of this document and acknowledgements}
This document started life as a collaborative Markdown document on
{\tt hackmd.io}, where contributions were also made by Raymond von
Bommel.  From there it migrated to a \Sage\ Jupyter notebook, which
included worked examples in both \Sage\ and~\Magma{}, before this
version.  The source for this document, as well as the \Sage\ Jupyter
notebook, are available at the public {\tt github} repository
\url{https://github.com/JohnCremona/BSD}.

Helpful contributions have been gratefully received from Tim
Dokchitser and Michael Stoll.

\begin{thebibliography}{99}
\bibitem{Dok}
  T. Dokchitser: \textit{Notes on the parity conjecture}
  \url{https://arxiv.org/abs/1009.5389}.
\bibitem{Gross} B. H. Gross, \textit{On the conjecture of Birch and
  Swinnerton-Dyer for elliptic curves with complex multiplication}. In
  Number theory related to Fermat’s last theorem (Cambridge, Mass.,
  1981), 219–236, Progr. Math., 26, Birkh¨auser. Boston, Mass., 1982.
\bibitem{MO}
  MathOverFlow\break
  \url{https://mathoverflow.net/questions/139575/bsd-conjecture-for-x-017}.
\bibitem{Silverman} J.H. Silverman, \textit{Computing heights on
  elliptic curves}, Math. Comp. {\bf 51} (1988), 339-358.
\bibitem{Tate}
  J. Tate: \textit{On the conjectures of Birch and Swinnerton-Dyer and a
  geometric Analog}. S\'eminaire N. Bourbaki, 1966, exp. no 306,
  p.~415-440.
\end{thebibliography}

\end{document}
