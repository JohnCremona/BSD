\documentclass{amsart}
\usepackage{amssymb}
\usepackage{amsmath}

\newcommand{\software}[1]{\textsf{#1}} % same as defi currently
\newcommand{\Sage}{\software{SageMath}{}\ }
\newcommand{\Magma}{\software{Magma}{}\ }
\newcommand{\GP}{\software{PARI/GP}{}\ }
\font\cyr=wncyr10
\newcommand{\Sha}{\text{\cyr X}}

\newcommand\Q{\mathbb{Q}}
\newcommand\R{\mathbb{R}}
\newcommand\C{\mathbb{C}}
\newcommand\hh{\hat{h}}

\begin{document}

\title{The BSD formula over number fields}
\author{John E. Cremona}
\maketitle

The Birch--Swinnerton-Dyer (BSD) formula for elliptic curves over \(\Q\)
is clearly stated and explained in many texts, but for elliptic curves
over general number fields there are a number of subtleties leading to
the appearance of factors which are equal to \(1\) in the rational case,
and clear and explicit statements are harder to find. We cite three (by
Tate, Gross, and T. Dokchitser) below, as well as a discussion on
MathOverflow highlighting the common confusions which people have. The
point of this text is to clear up the potentially confusing issues.

For simplicity we concentrate on the case of elliptic curves, only
briefly mentioning the adjustments needed for Abelian varieties of
higher dimension \(g\).

Traditionally, the BSD formula is written as a formula for the leading
coefficient of the L-series \(L(E,s)\) at \(s=1\), which we call the
\emph{L-value}, in terms of other quantities, including the order of
the \emph{Tate-Shafarevich group}~$\Sha$. We always assume the standard
conjectures, first that \(L(E,s)\) has analytic continuation so that
its behaviour at \(s=1\) is defined, and also that $\Sha$ is finite. Then
we rearrange the formula so that it expresses \(S=|\Sha|\) in terms of
quantities which may be computed.

\section{Code}\label{code}

Where possible, we give the appropriate commands in \Sage and \Magma to
compute these quantities. Many of these functions have a precision
parameter to control the precision of the output: the input is always
exact, but the output may in principle be computed to arbitrary
precision. \Magma can compute all the quantities separately (with a
little work) and can also evaluate the entire formula. \Sage can compute
all the quantities except for the L-value itself (unless \(K=\Q\)).

\subsection{\Magma}\label{magma}

For elliptic curves over general number fields all terms in the formula
are computed and combined in the function \texttt{ConjecturalSha()}:

\[
\texttt{ConjecturalSha(E,\ {[}P1,...,Pr{]});}
\]

\subsection{\Sage}\label{sage}

For elliptic curves over \(\Q\) only, we have:

\[
\texttt{E.sha().an\_numerical()}
\]

\section{Notation}\label{notation}

\(K\) is a number field of degree \(d=r_1+2r_2\), where \(r_1\) and
\(r_2\) are the number of real and complex places, and discrimninant
\(d_K\).

\(E\) is an elliptic curve defined over \(K\), given by an integral
Weierstrass equation or model with coefficients \(a_1,a_2,a_3,a_4,a_6\).
We do not assume that the model we have is minimal; global minimal
models do not always exist anyway. We denote by \(\Delta(E)\) the
discriminant of the model, which is an integral element of \(K\), and by
\(\mathfrak{d}(E)\) the minimal discriminant ideal of \(E\), an integral
ideal which is independent of the model. Then
\((\Delta(E))/\mathfrak{d}(E) = \mathfrak{u}^{12}\) for some fractional
ideal \(\mathfrak{u}\). (A global minimal model exists if and only if
\(\mathfrak{u}\) is principal; in general the class of \(\mathfrak{u}\)
is an invariant of \(E\).)

The Mordell-Weil group \(E(K)\) is a finitely-generated abelian group of
rank \(r=r(E(K))\), with finite torsion subgroup \(T=E(K)_{tors}\). Let
\(P_1,\dots,P_r\in E(K)\) be points which generate \(E(K)/T\).

\section{Outline formula}\label{outline-formula}

The formula for \(S\) has the form \[
S = \frac{(\text{Field factor})(\text{$L$-value})}{(\text{Mordell-Weil factor})(\text{Local factor})}
\]

\section{Field factor}\label{field-factor}

This is simply \[|d_K|^{1/2},\] so is equal to \(1\) for \(K=\Q\)).
Gross includes this in with the local factors at infinite places. For
Abelian varieties of dimension \(g\) the factor is \(|d_K|^{g/2}\).

\subsection{\Magma code:}\label{magma-1}

\[
\texttt{Sqrt(Abs(Discriminant(K)));}
\]

\subsection{\Sage code:}\label{sage-1}

\[
\texttt{RR(K.discriminant().abs()).sqrt()}
\]

\section{The L-value}\label{the-l-value}

This is \[L^{(r)}(E,1)/r!.\] Note that in general it is only conjectural
that the L-function of \(E\) has analytic continuation as far as
\(s=1\), so both the fact that this quantity is well-defined, and the
analytic properties assumed in the algorithm to evaluate its derivative
at \(s=1\), are conjectural.

\subsection{\Magma code:}\label{magma-2}

\[
\texttt{Evaluate(LSeries(E),1:\ Derivative:=r)\ /\ Factorial(r);}
\]

\subsection{\Sage code:}\label{sage-2}

For \(K=\Q\) only:

\texttt{E.lseries().dokchitser().derivative(1,r)\ /\ factorial(r)}

Otherwise on a system with \Magma installed one can resort to

\[
\texttt{RR(magma(E).LSeries().Evaluate(1,\ Derivative=r))\ /\ factorial(r)}
\]

\section{Mordell-Weil factor}\label{mordell-weil-factor}

This is \[\frac{R(P_1,\dots,P_r)}{|T|^2},\] where \(R()\) is the
regulator, i.e.~the determinant of the height pairing matrix. This needs
to be normalised correctly!

\begin{enumerate}
\item The \((i,j)\)-entry of the height pairing matrix is
  \[\left\langle P_i,P_j\right\rangle = \frac{1}{2}(\hh(P_i+P_j) - \hh(P_i) - \hh(P_j)),\]
  so in particular the diagonal entries are the canonical heights of the
\(P_i\);
\item The canonical height should be defined with respect to the
  \(x\)-coordinate (so that \(x(P)-\hh(P)\) is bounded independently
  of \(P\)). There is another convention (used in some of Silverman's
  papers) which is half of this, as it is defined with respect to the
  divisor \((O)\) instead of \(2(O)\). In the MathOverflow post,
  Silverman explains why it makes sense to use \(2(O)\) in the context
  of BSD.
\item The canonical height should \textbf{not} be normalized (by
  dividing by the degree \(d\)) to be invariant under
  base-change. Note that both \Magma and \Sage by default normalize
  the height. \Sage has an option to not normalize for heights, but
  not (as of version 9.1) for the regulator. Hence in both cases the
  regulator needs to be adjusted by multiplying by \(d^r\).
\end{enumerate}

\subsection{\Magma code:}\label{magma-3}

\[
\texttt{Regulator(P1,...,Pr)\ *\ d\^{}r\ /\ Order(TorsionSubgroup(E))\^{}2;}
\]

\subsection{\Sage code:}\label{sage-3}

\[
\texttt{E.regulator\_of\_points({[}P1,...,Pr{]})\ *\ d\^{}r\ /\ E.torsion\_order()\^{}2}
\]

The expression given here will only be correct if the points
\(P_1,\dots,P_r\) are \emph{saturated}, i.e.~generate all of \(E(K)\)
modulo torsion, so that
\([E(K):\left\langle P_1,\dots,P_r\right\rangle] = |T|\). An alternative
expression is sometimes seen which avoids this condition:
\[\frac{R(P_1,\dots,P_r)}{[E(K):\left\langle P_1,\dots,P_r\right\rangle]^2};\]
this is a neat trick, but not very useful in practice for computations.

\textbf{NB} If the points are not saturated but only generate a
subgroup (modulo torsion) of index \(n\), then this factor, and hence
the overall expression for \(S\), will be \(1/n^2\) times the correct
value. This will often lead to a non-integral value of \(S\),
which can be detected.

\subsection{Higher genus}\label{higher-genus}

For Abelian varieties \(A\) of general dimension \(g\) the formula
involves both \(A\) and its dual \(A'\), and points \(P_1',\dots,P_r'\)
generating \(A'(K)\) modulo torsion. (Recall that \(A\) and \(A'\) are
isogenous so have the same rank \(r\).) The factor is then
\[\frac{\det(\left\langle P_i,P'_j\right\rangle )}{\#A(K)_{tors}\cdot \#A'(K)_{tors}}\]
where now \(\left<\cdot,\cdot\right>\) is the height pairing between
\(A\) and \(A'\).

\section{Local factors}\label{local-factors}

The complete local factor is a product of individual local factors, one
for each place of \(K\). In our expresson the local factors at infinite
places will be defined in a way which depends on the model, so the
complete local factor also includes a correction factor making the
expression independent of the model.

\subsection{At finite places}\label{at-finite-places}

At a finite place \(v\) the local factor is \(c_v\), the Tamagawa number
at \(v\). This is a positive integer, the index
\(c_v = [E(K_v):E(K_v)^0]\), i.e.~the number of connected components of
\(E(K_v)\). At a place of good reduction we have \(c_v=1\), so this is
in effect a finite product.

\subsection{\Magma code:}\label{magma-4}

\[
\texttt{\&*TamagawaNumbers(E);}
\]
gives the product of the Tamagawa numbers \(c_v\).

\subsection{\Sage code:}\label{sage-4}

Over all number fields (including \(\Q\))
\[
\texttt{E.tamagawa\_product\_bsd()}
\]
gives the product of all Tamagawa numbers \(c_v\) times the correction
factor \(N(\mathfrak{u})\) (defined below) to allow for the model not
being minimal, while \texttt{E.tamagawa\_product()} returns
just the product of the Tamagawa numbers (over \(\Q\) only).

\subsection{At infinite places}\label{at-infinite-places}

At each infinite place the local factor is a suitably normalized period
\(\Omega_v\) and the total contribution from all infinite places is
\(\prod_{v\mid\infty}\Omega_v\). Theoretically the \(\Omega_v\) are
defined in terms of a Néron differential. We first define the periods
for a fixed model of \(E\) and then explain how to normalize it. Let
\(\omega_E = dx/(2y+a_1x+a_3)\) be the usual invariant differential
associated to a Weierstrass model for \(E\).

\subsection{At a real place}\label{at-a-real-place}

If \(v\) is a real place define \(\Omega_v(E)\) to be the integral of
\(\omega_E\) over \(E(\R)\). This is equal to the least positive real
period multiplied by the number of real components at the place \(v\)
(which is \(1\) when \(v(\Delta_E)<0\) and \(2\) when
\(v(\Delta_E)>0\)).

When \(v(\Delta_E)>0\) the period lattice of \(E\) at \(v\) has
generating periods \(x\), \(yi\) with \(x,y>0\), and \(\Omega_v=2x\).
When \(v(\Delta_E)<0\) the period lattice of \(E\) at \(v\) has
generating periods \(2x\), \(x+yi\) with \(x,y>0\), and \(\Omega_v=2x\).

\subsection{At a complex place}\label{at-a-complex-place}

If \(v\) is a complex place define \(\Omega_v(E)\) to be \textbf{twice}
the integral of \(\omega_E\wedge\overline{\omega_E}\) over \(E(\C)\). If
the period lattice of \(E\) at \(v\) has generating periods \(w_1\),
\(w_2\) with \(\Im(w_2/w_1)>0\), then
\(\Omega_v=2\Im(\overline{w_1}w_2)\) (which is positive).

If we were to define \(\Omega_v\) for complex \(v\) to be simply the
integral of \(\omega_E\) over \(E(\C)\), then the BSD formula would need
an additional factor of \(2^{r_2}\).

\subsection{Complete local factor}\label{complete-local-factor}

\[\frac{1}{\prod_{v\nmid\infty}c_v\prod_{v\mid\infty}\Omega_v N(\mathfrak{u})}.\]

\subsection{Adjustment for non-minimal models}\label{adjustment-for-non-minimal-models}

For a global minimal model, the minimal discriminant ideal is simply the
principal ideal \(\mathfrak{d}(E)=(\Delta(E))\). In general,
\((\Delta(E))=\mathfrak{u}^{12}\mathfrak{d}(E)\) for some fractional
ideal \(\mathfrak{u}\) whose valuation at each prime \(\mathfrak{p}\) is
\(1/12\) of the difference between the valuation at \(\mathfrak{p}\) of
\(\Delta(E)\) and that of the local minimal model at \(\mathfrak{p}\).

The total contribution from the infinite places is now
\[N(\mathfrak{u})\prod_{v\mid\infty}\Omega_v = \left|\frac{N(\Delta(E))}{N(\mathfrak{d}(E)}\right|^{1/12}\prod_{v\mid\infty}\Omega_v,\]
where \(N(\cdot)\) denotes the norm of a fractional ideal.

\subsection{\Magma code:}\label{magma-5}
\[
\texttt{Periods(E);}
\]
gives a basis for the period lattice with respect to each infinite
place, starting with the real places. In evaluating
\texttt{ConjecturalSha()} the product of the actual real and complex
periods \(\Omega_v\) are computed from these, including factors of 2 for
the real places where the discriminant is positive and a factor
\(2^{r_2}\), but the individual \(\Omega_v\) values are not accessible
to the user.

\subsection{\Sage code:}\label{sage-5}

\[
\texttt{{[}EK.period\_lattice(v).omega()\ for\ v\ in\ K.places(){]}}
\]
gives a list of the \(\Omega_v\) for all infinite places, real and
complex, including the factor of 2 for real places where the
discriminant is positive, but \textbf{not} the factor of 2 at complex
places. So to obtain \(\prod_{v\mid\infty}\Omega_v\) one needs the
following:
\[
\texttt{prod({[}EK.period\_lattice(v).omega()\ for\ v\ in\ K.places(){]})\ *\ 2**(K.signature(){[}1{]})}
\]

\section{The full formula}\label{the-full-formula}

\[
\Sha =
|d_K|^{1/2} \cdot \frac{L^{(r)}(E,1)}{r!} \cdot \frac{|T|^2}{R(P_1,\dots,P_r)} \cdot \frac{1}{N(\mathfrak{u}) \prod_{v\nmid\infty}c_v\prod_{v\mid\infty}\Omega_v}.\]

\section{Comparisons}\label{comparisons}

\subsection{Tate}\label{tate}

Tate has a factor \(|\mu|^g\) where \(g\) (which is \(d\) in Tate's
notation) is the dimension of the variety and he states that
\(|\mu|=|d_K|^{1/2}/2^{r_2}\). So this accounts for both our ``field
factor'' \(|d_K|^{1/2}\) and also the doubling of the complex periods.

\subsection{Gross}\label{gross}

Gross states the formula in the form L-value\(=M(A)R(A)h(A)\) where
\(h(A)=|\Sha|\). In the analogy with the analytic class
number formula, $\Sha$ plays the role of the class group whose order is
usually denoted \(h\). The other factors match our Mordell-Weil factor
and Local Factors respectively:

\(R(E)=R(P_1,\dots,R_r)/|T|^2\) for elliptic curves \(E\), adjusted as
before for general dimension.

\(M(A) = M_{\infty}(A)M_f(A)\), where (in our notation)
\[M_{\infty}(A) = \prod_{v\mid\infty}\Omega_v \cdot N(\mathfrak{u}) \cdot |d_K|^{-1/2},\]
and \(M_f(A)\) is the Tamagawa product \(\prod_{v\nmid\infty}c_v\).

\subsection{Dokchitser}\label{dokchitser}

On pages 3-5 he defines the regulator and L-value as we have. He says
that the regulator is the determinant of the \textbf{Néron-Tate
height-pairing} and the prefix ``Néron-Tate'' indicates that the height
is relative to \(K\) and not the normalized absolute one.

He combines all the local factors together into a quantity called
\(C_{E/K}\): Writing \(\omega\) for any invariant differential on \(E\)
(with arbitrary scaling), and \(\omega_v^o\) the local Néron
differential at a finite place \(v\), he defines \[
  C_{E/K} = \prod_{v\nmid\infty} c_v \left\vert\frac{\omega}{\omega_v^o}\right\vert_{_v}
    \>\cdot\>\prod_{{v|\infty},{\text{real}}} \int\limits_{E(K_v)}\!\! |\omega|
    \>\cdot\>\prod_{{v|\infty},{\text{cplx}}} 2\!\!\!\int\limits_{E(K_v)}\!\! \omega\wedge \bar\omega,
\] with \(c_v\) the local Tamagawa number at \(v\) and \(|\cdot|_v\) the
normalised absolute value on \(K_v\). If \(\omega\) is the differential
associated to a global minimal model then
\(\left\vert\frac{\omega}{\omega_v^o}\right\vert_{_v}=1\) for all finite
\(v\) and this reduces to our unadjusted local factor. In general the
factor \(N(\mathfrak{u})\) in our formula accounts for the difference in
scaling between the differential \(\omega_E\) and the local Néron
differential \(\omega_v^o\) at every place.

Note that \(C_{E/K}\) as defined here is independent of the model of
\(E\).

\section{References}\label{references}

\begin{enumerate}
\def\labelenumi{\arabic{enumi}.}
\item
  J. Tate: On the conjectures of Birch and Swinnerton-Dyer and a
  geometric Analog. Séminaire N. Bourbaki, 1966, exp. no 306,
  p.~415-440.
\item
  B. Gross
\item
  T. Dokchitser: Notes on the parity conjecture
  {[}https://arxiv.org/abs/1009.5389{]}
\item
  MathOverFlow
  {[}https://mathoverflow.net/questions/139575/bsd-conjecture-for-x-017{]}
\end{enumerate}
\end{document}
